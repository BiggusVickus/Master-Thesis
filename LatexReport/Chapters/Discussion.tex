\chapter{Discussion}
\label{Discussion}

\section{Graph Behavior}
\Cref{tab:results:graph_behavior} presents illustrative examples rather than a comprehensive analysis. 
The behaviors shown represent typical trends observed when varying each parameter in the specified direction, but they may not apply to all possible values or reflect the magnitude of changes. 
Additionally, these results do not necessarily generalize to scenarios where two parameters are varied simultaneously. 
\Cref{tab:results:graph_behavior} should be interpreted alongside the local S1 sensitivities from \nameref{sec:SOBOL_sensitivity_analysis_results} to better understand how sensitive the output is to specific parameters and the potential impact of their variation.

\section{A Good Curve}
As the bacterial population grows, resource consumption accelerates until only trace amounts remain at $t=8$. 
The delay between the peaks of uninfected and infected bacteria is due to the infection stages and the latent period of phage infection. 
Each bacterium progresses from infection stage $k$ to $k+1$ at a rate of $\frac{M}{\tau}$. 
Therefore, decreasing the number of infection steps $M$ or increasing the latent period $\tau$ amplifies this delay. 
A longer latent period means it takes more time for bacteria to progress through the infection stages.

At $t=4$, the infection rate surpasses the bacterial replication rate, causing the bacterial population to decline even though resources are still available. 
This moment coincides with the rise of the phage population. 
Observing the timing of these events and changes in graph behavior, as well as their relationships across different graphs, helps clarify the complex population dynamics and the interdependence of the populations that might not be obvious from reading the ODE model. 

This becomes more difficult when the model goes from a $1\times1\times1$ system to a $p\times b \times r$ system. 
Now up to any number of phages can interact with any number of bacteria, and any number of bacteria can interact with any number of resources at varying rates. 
These varying rates will significantly influence the dynamics of the system, and make it hard to determine what event caused what due to the rise in number of interactions.
For a $1\times1\times1$ system, there are 2 interactions that can occur (assuming no self interactions, and that phages don\t interact with the resources). 
With a $p\times b\times r$ system, there are $\mathcal{O}(p\cdot b + b\cdot r)$ interactions that can occur. 
So for a $3\times2\times3$ system, there are at most 12 interactions occurring. 
12 events can occur at the same time, making it hard to identify the cause of the event. 


\section{SOBOL Sensitivity}
\subsection{Final Value Analysis}
\subsubsection{Resources}
For most simulations, all resources no matter the initial value, will have been used up before the end of the simulation. 
So on average, the final value for resources will be 0. 
This explains why the resource sensitivity is so high. 
There are some combinations of parameters that when chosen, will result in not all the resources being used up. 
For example, with \Cref{tab:appendixE:a_good_curve_2}, if $r=0.2$, not all the resources have been used up. 
Every parameter affects how the bacteria and phage population grows in some way, which influences the consumption rate of resources, which determines the final resource value. 
Of the other parameters, $\tau$ has the biggest influence because $\tau$ determines how fast the bacteria will go through the infection process. 
The longer it takes, the longer it takes for the phage population to grow, allowing more bacteria to grow and consume resources. 

\subsubsection{Phages}
The $r$ value allows the phages to infect the uninfected bacteria. 
$r$, the adsorption rate of phages to bacteria, can be interpreted as the efficiency of infection. 
The smaller the value, the more efficient the infection process is, and fewer phages it requires to infect a bacterium. 
With a larger $r$ value, more phages are being used to infect a bacterium. 
$\beta$ still has an influence on the final phage population. 
The more phages that are created for every lysed bacterium, the more phages are available in the system. 
But large $r$ values will require more phages to infect 

\subsubsection{Total Bacteria}
Total bacteria is the sum of both uninfected and infected bacteria, so it makes sense for total bacteria to have similar values to uninfected and infected bacteria. 
Apparently the uninfected bacteria have a stronger influence on the output variance than the infected bacteria. 
The total bacteria sensitivities resemble the sensitivities of the uninfected bacteria more than the infected bacteria. 
It would have been expected for the total bacteria to resemble more of an average between the uninfected and infected. 

\subsection{Peak Value Analysis}
\subsubsection{Resources}

\subsubsection{Phages}
\subsubsection{Total Bacteria}

\subsection{Time of Peak Analysis}
\subsubsection{Resources}
\subsubsection{Phages}
\subsubsection{Total Bacteria}

\section{Initial Value Analysis}
The behavior between \Cref{fig:created:initial_value_analysis_UB_50_500_a_good_plot_2} and \Cref{fig:created:initial_value_analysis_UB_50_500_a_good_plot} should be similar, however the change in parameter values altered the simulation to introduce a region in behavior change. 
It would be expected that for 100 initial uninfected bacteria and less the bacteria sum peak time would follow the linear regression line, but at around 100 uninfected bacteria and less, the peak curve deviated from the linear expression. 
Obviously something in the model altering the phage and bacterial growth. 
The lack of resources is actually restricting the bacteria growth. 
Between 100 and 500 uninfected bacteria, the system is adsorption limited. 
There is a surplus of phages who in turn 
Between 50 and 100 uninfected bacteria, the system is burst limited. 
For uninfected bacteria less than 50, the system becomes resource limited. 
As the uninfected bacteria decreases from 50 towards 1, passing $K=10$, the bacteria growth rate start to slow down. 
This means that it takes longer for the bacteria to grow, as noticed by the increase in peak time relative to larger initial uninfected bacteria populations. 

As the phage population is driven by the bacteria population, with an increase in bacteria time of peak, there is also an increase in time of peak for the phages. 


\section{Phase Portrait}
There is non-linear tradeoff between initial resources and initial phages when there is washout included. 
The washout non-linearly affects if the phages proliferate or not. 
The higher the washout, the harder it would be for the phages to proliferate. 
Phage populations are coupled to bacteria populations which are coupled to resource populations. 
By varying the initial resource concentration, the bacteria growth rate is affected. 

For low initial resource concentration values, values below $10$ the resource, the monod curve is below the half-velocity constant (the velocity $v$ is 1 and $K$ is 10). 
The bacteria are restricted by the resources and can't grow quickly. 
So more phages are needed to grow. 
As N increases towards $K=10$, the bacteria can grow faster. 
Since $K$ is small, a small change in $R$ causes a relatively large change in the monod rate. 
This is noticed by the very steep downward line to the left of the minimum. 

At around the minimum, the behavior changes. 
At $R\equiv 6$, the monod rate reaches half of its max velocity. 
It should be 10, but the washout must have an effect on the monod rate, artificially shifting the point where the half velocity occurs from $K=10$ to $K=6$. 

As $R$ increases from $K$, the bacteria are not limited by the nutrients anymore and can grow faster. 
However, as $R$ increases beyond $K$, each additional unit of $R$ results in a diminishing increase in the Monod rate, which asymptotically approaches its maximum velocity. 
As bacteria grow according to the Monod equation $g(N, v, K)$, the phage population dynamics remain tightly coupled to bacterial growth. 

Phage proliferation becomes a race against time. 
If the phage growth rate is not fast enough to initially beat the washout removal rate, or the infected bacteria are washed out before lysing, the phages can not proliferate. 

\subsection{3D Plot}
It is not possible to see inside the matrix, but using the color on the outside can give some insights into the behavior happening inside the matrix. 
Even with the added bacteria, the phage proliferation boundary is still heavily dependent on the initial phages and initial resources, and less so on bacteria. 


\section{Plotting Parameter Change}
