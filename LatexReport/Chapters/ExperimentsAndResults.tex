\chapter{Experiments and Results}
\label{AER}
This section presents the experimental results obtained from investigating the interactions between bacteria and phages under varying nutrient levels. 
We first describe bacterial growth in the absence of phages across nutrient conditions to establish a baseline. 
Next, we report the effects of phage exposure on bacterial survival. 
Finally, we examine phage population changes over time. 
Data are presented as means ± standard deviations and supported by appropriate statistical analyses.

\section{Graph Behavior}
\Cref{tab:results:graph_behavior} elaborates how a change in parameter value changes the shape and population level of the agents. 
\nameref{sec:results:a_good_curve}, whose agent and parameter default values can be found in the first column of \Cref{tab:results:graph_behavior} and in \Cref{tab:appendixE:a_good_curve} are used as a reference graph. 
Each parameter was individually changed to a higher and lower value from the reference value, and the changes are noted in \Cref{tab:results:graph_behavior}. 

\begin{table}
    \small 
    \centering
    \begin{tabularx}{\textwidth}{l l X}
        \toprule
        \textbf{Parameter} & \textbf{Tested Value} & \textbf{Description of Behavior} \\
        \midrule
        $R$ (400) & 500 & More uninfected and infected, slightly more phages.\\
         & 300 & Slightly less uninfected and infected, earlier resource depletion.\\

        \midrule
        $U$ (50) & 70 & Slightly more phages and uninfected and infected bacteria.\\
         & 30 & Less uninfected and infected bacteria, slower resource depletion, not all resources used, slightly less phages. \\

        \midrule
        $P$ (10) & 20 & Less resources consumed, less uninfected, bacteria peaks earlier, slightly less phages.\\
         & 5 & Resources consumed faster, more uninfected, infected, and phages.\\

        \midrule
        $\tau$ (2.14) & 10 & Faster resource depletion, faster bacteria peak, plateau, then fall in population. more uninfected and infected, less phages.\\
         & 0.5& Barely any resource consumption, little bacteria growth and uninfected, more phages.\\

        \midrule
        $\omega^i$ (0) & 15 & Slightly more bacteria, resource replenish after bacteria die out.\\

        \midrule
        $e$ (0.03) & 0.1 & Faster resource depletion, sharper decline in uninfected, less infected and phages. \\
         & 0.01 & Less resource consumption, slightly more bacteria.\\

        \midrule
        $v$ (1.2) & 1.8 & More phages, significantly more bacteria, earlier and sharp peak in uninfected. \\
         & 1 & Less phages and bacteria, less resource consumption, earlier bacteria peak.\\

        \midrule
        $K$ (10) & 100 & Less resource consumption, less bacteria and phages, earlier bacteria peak.\\
         & 1 & Faster resource depletion and sudden stop instead of gradual slowdown, earlier bacteria peak.\\

        \midrule
        $r$ (0.01) & 0.1 & Less consumption, less infected and phages, earlier peak in bacteria..\\
         & 0.001 & Faster resource consumption rate, more infected and phages, delay in bacteria peak, sharp bacteria peak, small plateau in bacteria count before drop.\\

        \midrule
        $\beta$ (20) & 50 & More phages, earlier bacteria peak, less resources consumed, less bacteria.\\
         & 10 & Faster resource consumption, more uninfected, less phages, sharper bacteria peak.\\

        \midrule
        $\omega^o$ (0) & 0.02 & Faster resource depletion, more bacteria and sharper peak, later peak, and less phages. \\

        \bottomrule
    \end{tabularx}
    \caption{
        A table that compares how moving one individual parameter value up or down relative to the "\nameref{sec:results:a_good_curve}" changes the general shape of the curve. 
        This table is not meant to be exhaustive, cover edge cases, or extreme cases, or cover every exact detail and change in the population graph, but just to give an idea of how a change in parameter influences the graph shape, such as the rate of resource depletion, maximum number of bacteria and phages, and change in peak time. 
        Reference parameter values used to compare the produced curves are included in the parentheses, taken from \Cref{tab:appendixE:a_good_curve}. 
    }
    \label{tab:results:graph_behavior}
\end{table}


\section{A Good Curve}
\label{sec:results:a_good_curve}
\Cref{fig:created:a_good_curve_linear} shows an example of a good curve for a $1\times1\times1$ system. 
\Cref{fig:created:a_good_curve_logarithmic} is the same plot but with a logarithmic curve instead. 
The resources have been almost consumed by $t=8$, coinciding with the time at which the phages population starts to peak and which the bacteria population is quickly approaching 0. 
The uninfected and infected bacteria exhibit exponential growth, peaking at 1617 at $t=3.99$ and 3463 at $t=5.27$ respectively. 
The bacteria sum do not have as stark of a peak in comparison to the uninfected and infected bacteria, due to the graph measuring all bacteria populations, but the peak of 3805 at $t=4.89$ is still clear. 
The phages saw a significant increase in population count at around $t=4$, coinciding with the peak in uninfected bacteria. 
At around $t=4$ is when the the resource really start to deplete. 
At $t=5$ is when the resource consumption rate starts to slow down, showing a decreasing sigmoid shape. 
The total bacteria population reached a peak of 3805 at $t=4.89$, a 76.1x increase in population count from the initial 50 starting uninfected bacteria. 
The phage population reached a peak of 2584 phages at $t=15$, a 258.4x increase in population count. 
\begin{figure}[h!]
    \centering
    \begin{subfigure}{1\linewidth}
        \centering
        \captionsetup{width=1\linewidth}
        \includegraphics[width=\linewidth]{Plots/Created/a_good_curve_linear.png}
        \caption{
            Linear y-axis for a "good" plot. 
        }
        \label{fig:created:a_good_curve_linear}
    \end{subfigure}
    \hfill
    \begin{subfigure}{1\linewidth}
        \centering
        \captionsetup{width=1\linewidth}
        \includegraphics[width=\linewidth]{Plots/Created/a_good_curve_logarithmic.png}
        \caption{
            Logarithmic y-axis for a "good" plot. 
        }
        \label{fig:created:a_good_curve_logarithmic}
    \end{subfigure}
    \caption{
        Growth population of a $1\times1\times1$ system. 
        The log plot allows to see behavior happening at values approaching and to plot data on a logarithmic scale. 
        The parameters used for this plot can be found in \Cref{tab:appendixE:a_good_curve}. 
    }
    \label{fig:created:a_good_curve}
\end{figure}

\section{SOBOL Sensitivity Analysis Results}
\label{sec:SOBOL_sensitivity_analysis_results}
It is important to understand how a change in parameter value affects the change in output of a model. 
Models will have parameters that are more important and have a larger effect on the model output than other parameters. 

\Cref{fig:created:SOBOL_final} shows the impact that the parameter had on the final value of the population at $t=15$. 
THe average value and variance of population value were intentionally left out of the analysis, despite being a part of the dashboard because the SOBOL sensitivity values are almost identical to the final sensitivity values. 
There were some very minor differences from bar to bar across plots, but the difference was imperceptible. 
Since the plots all look similar, only an analysis on the final value, \Cref{fig:created:SOBOL_final} will be done. 

The parameters that were tested include all the parameters listed in the extended golden model, except for Uninfected Bacteria and $M$. 
Uninfected Bacteria was left out as it doesn't make sense to already add infected bacteria to the system
$M$, the number of stages that the infection goes through, can not be tested as $M$ hardcodes the number of infection stages that the bacteria has to go through. 
The hardcoding is done before the simulation framework starts. 
As such, it is not possible to change $M$ without rerunning the program from the very start. 

\subsection{Final Value Analysis}
\subsubsection{Resources}
The $\omega^i$/washin rate had the largest influence on the final, average and variance value. 
Without a washin rate, the resources will most likely have been consumed by the time the simulation ended at $t=15$. 
The final values for Resources, Uninfected, Infected, and Phages would often be something similar to (0, 0, 0, 10000) at $t=15$, where all the resources were consumed and the bacteria died out due to the phages, leaving only the phages remaining. 
The final value of the resources would often be 0, no matter what parameter values were used, with $\omega^i, \omega^o = 0$. 
With the addition of the washin, new resources were constantly being re-added. 
Once the bacteria died out, the resources could accumulate, with the accumulation dependent on the rate of the washin rate, hence why the washin rate has such a large impact on the final, average, and variance of population value for the resources. 
The final value would be dependent on when the bacteria died out, in turn allowing the resources to accumulate at a rate proportional to $\omega^i - \omega^0\cdot R$. 
Resources were less dependent on higher order interactions, unlike the uninfected, infected, phages, and total bacteria sum. 

\subsubsection{Uninfected}
The uninfected bacteria population sensitivities depend on many higher order interactions between the parameters as $ST_i \gg S1_i$. 
The uninfected are highly dependent on $\beta$/B\_matrix and initial phage population, as the initial phage population will determine how many bacteria become infected, and how quickly the phages can proliferate through the bacteria population. 
Surprisingly, $r$/r\_matrix did not have as big of an influence on the uninfected as $\beta$ did, even though the infection rate is dependent on $r$. 
The larger or smaller $r$ is, the faster or slower the infection rate is. If $r$ is really small, the infection rate would take forever, potentially allowing the bacteria to keep a stable population. 
$r$ is equally as important at explaining the final value as $\tau$/tau\_vector, washin, $e$/e\_matrix, and washout of sensitivity around 0.25. 

\subsubsection{Infected}
Since $ST_i \gg S1_i$ for the infected bacteria, where $S1_i \approx 0$ for nearly all of the parameters, the infected bacteria heavily depend on many interactions happening at the same time. 
This makes intuitive sense after looking at \nameref{sec:golden_model}. 
The infected (and uninfected) bacteria directly interact with $R$, $U$, $P$, $v$, $K$, $r$, $\tau$, and $\omega^o$ ($M$ is not included as it was left out of the analysis). 
So due to the high coupling of parameters, the infected (and also the uninfected) have large global sensitivities compared to the local sensitivity. 

However SOBOL had some difficulties assigning a good sensitivity score to each parameter for the $ST$ and $S1$ tests as noticed by the slightly larger error bars in the infected than the uninfected or resources. 
This is most likely due to the infected bacteria going through multiple stages of infection, causing a delay and uncertain behavior in the final value, despite the ODE model being deterministic. 

\subsubsection{Phages}
The most important factor for the final phage value is $r$, followed by $\beta$, $\omega^o$ and $P$. 
The $r$ value allows the phages to infect the uninfected. 
When $r$ is decreased, the final phage population is counterintuitively higher than when $r$ is larger. 
The behavior is counterintuitive because one would expect that a higher infection rate would lead to more infections and thus more phages. 
With a higher $r$ value, more phages are being removed from the phage population and infecting the bacteria. 
It can be seen as a way of "more phages are needed to infect a bacterium", therefore getting less phages out as a result as more phages are needed to infect a single bacteria. 

Washout has a noticeable influence on the phage population, as not the phage population is being reduced at a rate proportional to the washout rate. 
The larger the washout rate, the larger the drawdown of phages. 
When the infected all die out, the phage population wont grow anymore. 
Given the phage population at that point in time, the phage removal rate is proportional to the washout rate. 

\subsubsection{Total Bacteria}
Total bacteria is the sum of both uninfected and infected bacteria, so it makes sense for total bacteria to have similar values to uninfected and infected bacteria. 
Apparently the uninfected bacteria have a stronger influence on the output variance than the infected bacteria. 
The total bacteria sensitivities resemble the sensitivities of the uninfected bacteria more than the infected bacteria. 
It would have been expected for the total bacteria to resemble more of an average between the uninfected and infected. 

\begin{figure}[h!]
    \centering
    \begin{subfigure}{0.32\linewidth}
        \centering
        \captionsetup{width=1\linewidth}
        \includegraphics[width=\linewidth]{Plots/Created/SOBOL/SOBOL_analysis_1748084143_Final.png}
        \caption{
            Final population value. 
        }
        \label{fig:created:SOBOL_final}
    \end{subfigure}
    \hfill
    \begin{subfigure}{0.32\linewidth}
        \centering
        \captionsetup{width=1\linewidth}
        \includegraphics[width=\linewidth]{Plots/Created/SOBOL/SOBOL_analysis_1748084143_Peak.png}
        \caption{
            Peak population value. 
        }
        \label{fig:created:SOBOL_peak}
    \end{subfigure}
    \hfill
    \begin{subfigure}{0.32\linewidth}
        \centering
        \captionsetup{width=1\linewidth}
        \includegraphics[width=\linewidth]{Plots/Created/SOBOL/SOBOL_analysis_1748084143_Peak_Time.png}
        \caption{
            Time at peak population. 
        }
        \label{fig:created:SOBOL_peak_time}
    \end{subfigure}
    \caption{
        SOBOL analyses for the average, peak, and peak time. 
        The data was saved from the dashboard and plotted using Matplotlib. 
        The average and variance analysis results were left out for nearly identical results to the final value. 
        The values used for this SOBOL test can be found in \Cref{tab:appendixE:SOBOL_analysis_values}. 
        The data used in \Cref{fig:created:SOBOL_final} was used for \Cref{fig:created:SOBOL_peak} and \Cref{fig:created:SOBOL_peak_time}. 
        The plot of the average and variance analysis can be found at \Cref{fig:created:SOBOL_average_extra} and \Cref{fig:created:SOBOL_variance_extra}
    }
    \label{fig:created:SOBOL_analyses}
\end{figure}

\subsection{Custom SOBOL Analysis - Peak Value and Peak Time}
Due to the similarity of the final, average, and variance value as seen in \Cref{fig:created:SOBOL_final}, \Cref{fig:created:SOBOL_average_extra}, and \Cref{fig:created:SOBOL_variance_extra} a custom SOBOL analysis that isn't included in the dashboard might result in a different SOBOL analysis result. 
To create the custom SOBOL analyses, the peak value and the time at the peak of the population is measured and analyzed. 
The peak is defined as the point where the population reaches 95\% of its absolute maximum value. 
The time at peak is measured at the point in time that the population reaches 95\% of the maximum value. 
This removes unintended side effects of the simulation. 
For populations that are only increasing in value, this prevents the measured peak from bunching up at the end of the simulation, skewing the data. 
As the peak is defined at 95\% of the absolute maximum value, populations that have a faster increase on population count at the end will have a time value closer towards the end of the simulation. 
For populations that reach a plateau, the 95\% rule will push the peak time towards the beginning of the simulation, while still "respecting" the absolute final value since $95\% \approx 100\%$. 
The 95\% rule can fail under certain situations, such as when there is cyclic behavior. 
See \nameref{sec:appendixF:why_95} for a more detailed explanation on why the 95\% rule is used. 

The results of the SOBOL peak and time at peak analyses can be seen in \Cref{fig:created:SOBOL_peak} and \Cref{fig:created:SOBOL_peak_time}. 
Although some of the bars between the final, peak, and time at peak values are the same, some are different. 
But overall, similar values can be seen across the the final, peak, and time at peak analyses. 
The peak infected values are more certain compared to the final infected values, which could be due to the 95\% rule removing some of the noise of the simulation. 
The time at peak values have less error compared to the final and peak value. 
This is due to the restricted range of values. The time at peak value can only fall somewhere between 0 and 15, the start and end values of the simulation respectively. 
The final and peak values can fall anywhere between 0 and any value, depending on the IC and how high the population can rise, and how fast the population can fall, \textit{if} the population count falls. 

\subsection{SOBOL Analysis - Without Washin and Washout}
In many of the plots, the washin and washout rate had a large influence on the final, peak, and time at peak value. 
\Cref{fig:created:SOBOL_no_wi_wo_extra} ran the same input as \Cref{fig:created:SOBOL_analyses}, but left the washin and washout rate out. 
The sensitivity plots for the final, average, variance, peak, and time at peak plots look different form one another. 
The final resource value depended heavily on the washin and washout rate, but without the washin and washout, the final resource depended heavily on the initial resource population. 
Since $S1 \approx ST$, the resource parameter does not depend on other higher order interactions. 

The peak value for the resources without washin and washout only depended on the initial resource consumption. 
Since there was no washin, no resources could be added, so the peak for the resources was always at $t=0$, and was dependent on the initial resource value. 
The time at peak value is always 0 as the resources are only being depleted, so no matter the change in parameter values, the parameter had no effect on the peak time, so SOBOL gives a value of 0 to every parameter for the resources. 
$\beta$ consistently had a large effect on the final, average, variance, peak, and time of peak value as 


\section{Initial Value Analysis Results}
\label{sec:results:initial_value_analysis}

The IVA section of the dashboard allows the user to visualize how a change in parameter value affects the population growth of the agents. 

In Figure 1 of \citet{mullaExtremeDiversityPhage2024}, they vary the initial concentration of bacteria and measure the time until bacterial collapse. 
The initial concentration value and time of collapse is plotted on the x and y-axis with a tight linear regression fit on the data. 
The observed logarithmic decrease suggests that the phage kinetics is adsorption-limited meaning that 
\Cref{fig:created:initial_value_analysis_UB_50_500_a_good_plot_2} replicates this graph. 

\Cref{fig:created:initial_value_analysis_UB_50_500_a_good_plot}, even considered a "good" curve, shows interesting behavior that diverges from that of \Cref{fig:created:initial_value_analysis_UB_50_500_a_good_plot_2}. 
Then though the behavior should be similar, changing other parameter values can alter how a parameter works. 
It would be expected that for 100 initial uninfected bacteria and less the bacteria sum peak time would follow the linear regression line, but at around 100 uninfected bacteria and less, the peak curve goes horizontal. 
This suggests that something is restricting the bacterial growth. 
The lack of resources is actually restricting the bacteria growth. 

\begin{figure}
    \centering
    \begin{subfigure}{1\linewidth}
        \centering
        \includegraphics[width=\linewidth]{Plots/Created/IVA/initial_value_analysis_UB_50_500_a_good_plot.png}
        \caption{
            IVA for \Cref{tab:appendixE:a_good_curve}. 
            Replicates Figure 1 of \citet{mullaExtremeDiversityPhage2024}. 
        }
        \label{fig:created:initial_value_analysis_UB_50_500_a_good_plot}
    \end{subfigure}
    \hfill
    \begin{subfigure}{1\linewidth}
        \centering
        \includegraphics[width=\linewidth]{Plots/Created/IVA/initial_value_analysis_UB_50_500_a_good_plot_2.png}
        \caption{
            IVA for \Cref{tab:appendixE:a_good_curve_2}. 
            For uninfected bacteria less than 100, the phage-bacteria interaction is resource limited. 
            For 100 and higher, it is adsorption limited. 
        }
        \label{fig:created:initial_value_analysis_UB_50_500_a_good_plot_2}
    \end{subfigure}
    \caption{
        Varying initial Uninfected Bacteria concentration, from 50 to 500, with 30 unique values tested over two different instances of "good" curves. 
        Even with two "good" curves, even varying the default parameter values a tiny bit can have a large influence on how changing the initial bacteria concentration can have an influence on the dynamics of the system, changing the behavior of the peak time. 
        The default values for Figure a) and b) can be found at \Cref{tab:appendixE:a_good_curve} and \Cref{tab:appendixE:a_good_curve_2}. 
    }
\end{figure}


\section{Phase Portrait}
\label{sec:results:phase_portrait}
\Cref{fig:created:phase_portrait_resources_245-265_phages_25-26} shows a phase portrait varying the initial resource and phage concentration. 
For phages that start above 25.98, the phage population can proliferate (until the washout would eventually removes the phages). 
For phage populations that start below 25.98, the washout removes the phages before the phages had time to infect and kill the bacteria. 
Both regions of phages exhibit consistent behavior, of either going to 0 or proliferating. 
If the phage population started at exactly 25.98, if the initial resources was 260 or above, the phages died out. 
If the initial resources was 255 or below, the phages would proliferate. 

\Cref{fig:created:phase_portrait_resources_phage} simulates a wider range of values. 
The initial resource values span from 100 to 500, and the initial phage values range from 25.5 to 26.5, each with 100 unique values sampled.
It shows if a set of initial conditions allowed the phages to proliferate (red boxes) or if the phages died (white boxes). 
A boundary between the dead and proliferating phages can be curve-fit. 
The curve follows a logarithmic $y=21.484\cdot ln(x+3.353)-21.810$ with an $R^2$ value of 0.994. 
There appears to be a non-linear tradeoff between initial resources and phages when there is washout. 
The washout non-linearly affects if the phages proliferate or not. 

Phage populations are coupled to bacteria populations hwo are in turn coupled to resource populations. 
By varying the initial resource concentration, the bacteria growth rate is affected, in turn affecting the phage population, with a non-linear effect. 



\begin{figure}[]
    \centering
    \begin{subfigure}{0.49\linewidth}
        \centering
        \includegraphics[width=1\textwidth]{Plots/Created/PP/phase_portrait_resources_245-265_phages_25-26.png}
        \caption{
            Zoomed in plot of a phase portrait with varying resource and phage population from 40-60 and 60-70 respectively. 
            Each row has its own line color. 
            Diverging behavior can be seen for the orange line (phage=25.98). 
        }
        \label{fig:created:phase_portrait_resources_245-265_phages_25-26}
    \end{subfigure}
    \hfill
    \begin{subfigure}{0.49\linewidth}
        \centering
        \includegraphics[width=\linewidth]{Plots/Created/PP/phase_portrait_resources_phage.png}
        \caption{
            Phage population proliferation as a function of initial resource and phage concentrations. 
            While the color appears uniform along the vertical axis, each cell is actually a slightly different value. 
            The phage-resource proliferation boundary follows a fitted Monod equation.
        }
        \label{fig:created:phase_portrait_resources_phage}
    \end{subfigure}
    \hfill
    \begin{subfigure}{0.49\linewidth}
        \centering
        \includegraphics[width=\linewidth]{Plots/Created/PP/phase_portrait_resources_phage_2.png}
        \caption{
            Zoomed in to analyze the regime of behavior change near resources$=10$. 
        }
        \label{fig:created:phase_portrait_resources_phage_2}
    \end{subfigure}
    \caption{
        Varying initial resources and initial phages and the resulting curve and proliferation. 
        The clear phage proliferation boundary follows a shifted Monod curve. 
        Proliferation is defined as when the phage population reached at least 2 times the initial starting population. 
        This simulation values used can be found in \Cref{tab:appendixE:a_good_curve_2}, but with washout set to 0.02 instead of 0. 
    }
    \label{fig:created:phase_portrait_resource_phage_proliferate}
\end{figure}


\section{Plotting Parameter Change - $3\times 2\times 3$ Model}
\Cref{fig:created:r_beta_washout_0}, \Cref{fig:created:r_beta_washout_0.02}, and \Cref{fig:created:r_beta_washout_0.05}, show a 7x7 matrix of different $r$ and $\beta$ initial conditions for a $3\times2\times3$ model, and each figure changes the washout rate from 0 to 0.02 to 0.05. 
For each simulation, if $r$ or $\beta$ is equal to a value not equal to $inf$, as identified by the title above each sub-figure, then each value of $r$ or $\beta$ has that value.
All phage initial values started at 10. This was specifically chosen 
If $r$ or $\beta$ is equal to $inf$, then the simulation uses the original data as defined in the IC, vector, and matrix section of the dashboard. 
As a small example for the $3\times 2\times 3$ model, when $r=0.200$, the simulation framework uses $r = \left(\begin{smallmatrix} NaN & 0.200 \\ 0.200 & NaN \\ 0.200 & 0.200 \\ \end{smallmatrix}\right)$ as the input matrix to the ODE model. 
When $r=inf$, the simulation framework uses $r=\left(\begin{smallmatrix} NaN & 0.11695 \\ 0.144459 & NaN \\ 0.11895 & 0.13065 \\ \end{smallmatrix}\right)$ as the data to simulate the interactions with. 
For the cells that don't have an edge between $p$ and $b$ or between $b$ and $r$, the data is represented as $NaN$, short for "Not a Number", \textit{np.NaN}, 

The columns and rows of each figure makes it easy to compare how a change in parameter value affects the curve, while keeping the other parameter the same. 
In $r, \beta, \omega^o=inf, inf, 0$, although not a "good curve" due to limited resource consumption and limited bacterial and phage growth, really shows how the different parameter values for each interaction uniquely affect the growth rate of each agent, especially the phage population (P0=blue, P1=green, and P2=purple). 
Despite all phages starting at the same population level, within the first two or so time units, P1 has less phages than P0 and P2. 
P2 has the fastest initial growth rate, as P2 has the most phages until $t=4$, at which point P1 has a larger phage population. 
P2 reaches its peak population count before P0 or P1, but despite the slower initial growth, P0 and P1 eventually overtake P2 in total phage population. 
P2 also actually reaches its peak before decreasing in population. 
There are trace amounts of infected bacteria at the end of the simulation. 
Since the phage population is reduced by $r_{pb}\cdot(U_b + \sum_{k=1}^M I_{b_k})$, and by specifically choosing the parameter values as used in \Cref{tab:appendixE:complex_model}, behavior that hasn't been seen in a $1\times 1\times 1$ system has been found. 
The complete extinction of the bacteria has been delayed long enough such that at trace amounts, there is phage reduction despite bacteria still existing. 
The peak times for P0, P1, and P2 are $t=6.33, 7.99, 4.52$, a difference of $3.47$ time units. 

Contrast that with the phage population dynamics of that with $r, \beta, \omega^o = inf, 100, 0$, the phage populations do not show interesting dynamics. 
The peak times are more similar and consistent to one another ($t=5.50, 7.01, 6.78$, a difference of $1.51$ time units). 
The phage population curve all appear the same, with slightly slower growth rates. 
There is no crossing of phage populations unlike with $r, \beta, \omega^o=inf, inf, 0$. 

Losing the change in $\beta$ values really affected the dynamics fo the phage population. 
The highlighted example demonstrated the dynamics and influence that multiple agents can have on the final output. 

The top row of \Cref{fig:created:r_beta_washout_0.02} shows how the phages and resources died out relative to the top row of \Cref{fig:created:r_beta_washout_0}. 
Even with a high burst value, the phages could not defeat the pressure from the washout. 
But by changing the $r$ value from $r=0.001$ to $r=0.041$, the phages were able to save themselves and proliferate. 
Using this knowledge, and the information gained from \nameref{sec:results:phase_portrait}, a 3D matrix of proliferation could theoretically be created and displayed. 

\begin{figure}[]
    \centering
    \begin{subfigure}{0.32\linewidth}
        \centering
        \includegraphics[width=\linewidth]{Images/Plots/Created/UA/r_beta_washout_inf_inf_0.png}
        \caption{
            $r, \beta, \omega^o = inf, inf, 0$, all initial phage condition=10. 
        }
        \label{fig:created:r_beta_washout_inf_inf_0}
    \end{subfigure}
    \begin{subfigure}{0.32\linewidth}
        \centering
        \includegraphics[width=\linewidth]{Images/Plots/Created/UA/r_beta_washout_inf_100_0.png}
        \caption{
            $r, \beta, \omega^o = inf, 100, 0$, all initial phage condition=10. 
        }
        \label{fig:created:r_beta_washout_inf_100_0}
    \end{subfigure}
    \begin{subfigure}{0.32\linewidth}
        \centering
        \includegraphics[width=\linewidth]{Images/Plots/Created/UA/r_beta_washout_inf_inf_0.02.png}
        \caption{
            $r, \beta, \omega^o = inf, inf, 0.02$, all initial phage condition=10. 
        }
        \label{fig:created:r_beta_washout_inf_inf_0.02}
    \end{subfigure}
    \caption{
        Varying $r$, $\beta$, and $\omega^o$. 
        All phage values set to 10 to show how the network connections and vector/matrix values affect phage growth. 
        Selectively chosen sub-figures from \Cref{fig:created:r_beta_washout_0}, \Cref{fig:created:r_beta_washout_0.02}, and \Cref{fig:created:r_beta_washout_0.05}. 
        Chosen parameter values can be found in \Cref{tab:appendixE:complex_model}. 
    }
\end{figure}