\chapter{Conclusion and Future Work}
\label{CAFW}

\section{Conclusion}
\label{Conclusion}
Phages have traditionally been viewed as the enemy of bacteria. 
But with the advent of new transcriptomic, proteomic, genomic, and metabolomic methods, we have gained a better understanding of how phages influence bacterial populations in ecosystems, as well as in laboratories. 
Data from transcriptomics, proteomics, and metabolomics studies have shown how bacterial responses to phages vary and alter the phenotype of the bacterial host. 
Data suggests that phages have a net positive effect on bacterial populations. 
More research is needed to understand how phages and bacteria can coexist, as well as how the removal of a phage, bacteria, or their interactions can influence the dynamics of microbial communities \cite{fernandezPhageFoeInsight2018}. 

Understanding the relationship between phages and bacteria, as well as their interaction with the environment, is complex. 
For a single phage-bacteria pair growing on one resource, there are seven biological parameters $(e, v, K, r, \beta, M, \tau)$ inputs for the Golding model, and six non-biological parameters ($R, U, I, P, \omega^o, \omega^o$). 
Although relatively small in comparison to other models, analyzing 13 unique parameters and their interactions takes time and requires an intricate understanding of the model. 
Finding parameter values that result in high-quality and noteworthy graphs is not an easy task, although knowing the expected values and their biological relevance makes the task easier. 
Finding a set of parameters that yields behavior worth analyzing takes time. 

\subsection{Simulation Framework}
To help aid myself with the task, I created a simulation framework that anyone can use to analyze their own custom model and interaction network. 
Users can visually create and edit their interaction networks. 
Using the dashboard, they can edit the parameter values and run their simulations. 
The dashboard includes prebuilt visualization tools that enable users to interact with the simulation results. 
The tool enables users to quickly iterate over and modify parameter values to observe their impact on simulation results. 
Although the visualizations on the dashboard are designed explicitly for $1\times 1\times 1$ models, users can download the complete simulation data and implement their visualization methods to visualize the data over a $p\times b\times r$ system. 
I use this tool to significant effect in introducing visualization methods, such as the IVA, to identify growth bottlenecks. 
I take the analysis further by introducing new visualizations, such as a phage proliferation analysis and a survivability matrix. 

Using the default interactive graphs on the dashboard and their custom visualizations, users can explore the parameter space to identify notable model behaviors and gain a deeper understanding of the system’s dynamics. 
As evident by \Cref{fig:created:initial_value_analysis_UB_50_500_a_good_plot} and \Cref{fig:created:initial_value_analysis_UB_50_500_a_good_plot_2}, different parameter values will lead to contrasting behavior, even if both parameter models replicate realistic growth curves. 

\subsection{Larger Systems}
Trying to expand an analysis into a $p\times b\times r$ system becomes even more complicated due to the interconnected nature of the entities. 
With increasingly larger systems, small changes in a single parameter will often have a minimal influence on the final output. 
If the parameter value has a substantial influence, it can have a cascading effect on the entire network. 
As an example, with a $2\times 2\times 1$ system, where $P_1$ infects $B_1$, and $P_2$ infects $B_1$ and $B_2$, and both bacteria consume $R_1$.  
Increasing the infection rate of $P_1$ will slow the growth of $B_1$ as $B_1$ is infected more rapidly. 
With slower $B_1$ growth and less uninfected $B_1$, $P_2$ is affected as there are fewer $B_1$ to infect. With a lower $P_2$ count, $B_2$ can grow, using more resources. 
As there are now fewer resources, it is harder for $B_1$ to grow, so $B_2$ can grow. 
Eventually, $P_2$ starts to infect $B_2$, so $B_2$ starts to die, which gives $B_1$ a chance to consume resources and grow. 
A self-reinforcing feedback loop starts, where a change in the infection rate has a cascading effect on the rest of the network. 
The coupled interactions will form a feedback loop, resulting in non-obvious behavior. 

\subsection{Parameterizing Matrices}
Parameterizing complex systems is a challenging task. 
Parameterizing high-dimensional systems can be challenging. 
In a $p \times b \times r$ model with $p, b, r > 1$, setting a parameter (e.g., $\beta$ to 35) can be approached in several ways: 1) assign 35 to every element in the $p \times b$ matrix, 2) randomize values such that their mean is 35, or 3) scale existing values to achieve a mean of 35.

Typically, for bacterial communities, a common choice is to use a random parameter matrix, possibly with a predetermined structure. 

There are numerous interactions and parameters in large and complex systems, making it challenging to analyze them effectively. 
The model network and parameter values are relatively random, and a figure of the network interactions, along with a copy of the parameter inputs, are needed to analyze why a phage or bacterium is behaving in a certain way. 
Furthermore, it is challenging to simplify or eliminate a phage or bacterium from the model due to the interconnected nature of the system. 

\section{Future Work}
\label{Future Work}
The following steps involve collaborating with the researchers running the lab experiments to verify the results, as seen in the output, by comparing the lab results with the model output. 
With the lab results, the model can be adjusted to better align with the lab findings. 
The ODE results can replicate the lab results by modifying parameter values or adjusting the model equation. 
The user can decide to add the Monod microbial growth model to the growth of the bacteria or adapt the Monod equation to being dependent on multiple sources. 
Using the model, the technicians can improve and validate their methods. 
If the empirical results significantly deviate from the model results, the technician can theorize what might be happening and alter the model to account for the discrepancy. 
\citet{deyEmergentHigherorderInteractions2025} was able to adapt their model to account for the discrepancy between the model results and the results seen in the lab. 
They theorized that phages were somehow being deactivated. 
By adding the debris term, they were better able to account for phage deactivation and achieved a better, more accurate curve fit. 

\subsection{Model Replication}
Being able to replicate other models like that of \citet{nilssonCocktailComputerProgram2022} would allow me to compare model outputs. 
A benefit of implementing Cocktail’s model is that it would be possible to model multiple bacteria and phages at the same time, as noted as a limitation in \Cref{sec:literature:cocktail_and_phagedyn_limitations}. 
Cocktail limits itself to two phages, one bacteria, and one resource. 
Cocktail supports adding more phages at set times, but only at most three times. 
This arbitrary limitation can be removed with Cocktail’s model implementation. 

\subsection{Debris}
Further investigation into the debris and its effects could form the next step in the project.
I demonstrated that adding a debris term increased the survivability of bacterial populations on average, resulting in a higher uninfected bacterial population and lower counts of both uninfected and phage populations.

\subsection{Lab Work}
The next logical step is to conduct lab work to generate curves for comparison with the simulation results. 
If these empirical curves differ significantly from those produced by the current ODE model, a new ODE model can be developed. 
Curve fitting algorithms can then be employed to estimate the interaction parameters for the revised model numerically. 
By leveraging the simulation software, researchers can reduce the number of required experiments, thereby saving time, money, and resources.

The lab work would act as an essential model validation step. 
\citet{deyEmergentHigherorderInteractions2025} showed how their ODE model would eventually diverge from the lab-produced ODE curve. 
They were able to achieve a better curve fit by adapting the model to include the debris term. 

Future lab work can also involve identifying the bacteria, phages, and resources present in marine water by analyzing samples collected from the environment. 
The next step would be to build an interaction network, along with experimentally determining the interaction parameter values through laboratory work. 
Researchers can predict how the system would behave under new, untested conditions, saving money and time. 
It may also indicate to researchers if they made an error during testing. 
Suppose the model indicates that the system should behave in one way, but the system acts differently. In that case, the researcher can review their methods and consider making adjustments to how they conduct the experiments. 
All in all, having a model that runs in seconds will help researchers gain a better understanding of the system. 

\subsubsection{Environmental Modelling}
Many results in research papers come from controlled lab settings. 
As a next step, researchers can actively collect daily water samples and measure phage, bacteria, and resource concentrations. 
Collecting samples for over a year would create an ODE-like population curve of the entities. 
This approach would provide deeper insights into the dynamics of bacterial and phage populations in natural environments, albeit at the expense of losing control over conditions. 
By continuously monitoring environmental factors such as hourly temperature, rainfall, and the concentrations of each entity, researchers can gain a deeper understanding of the causal relationships within the ecosystem.
A year of experimentation averages out short-term fluctuations in daily measurements, resulting in a smoother overall curve.

The next step would be to use the model to fit and explain observed population dynamics. 
We want to model how environmental variables (such as temperature, nutrient availability, and rainfall) influence the interactions between phages, bacteria, and resources. 
Key phenomena could include year-long seasonal cycles, such as dry and wet seasons, rapid growth and decline in population counts, resilience, and the impact of random events, including storms or pollution spikes. 
By fitting the model to real-world data, we could identify which parameters or interactions are most sensitive to environmental changes and predict how the ecosystem might respond to future events and scenarios. 
Trying to isolate these communities and run different experiments could be the next step. 
Analyzing which phages interact with which bacteria or conducting a knockout experiment to investigate how the loss of a bacterium or resource node has a cascading effect on population growth. 

\citet{cleggCrossfeedingCreatesTipping2025} created a $b\times r$ bacteria-resource model without phages and identified which bacteria consumed which resources. 
By adjusting the number of resources required for survival, the researchers were able to alter the community’s diversity. 
By adding and removing bacteria-resource interactions to the generalized Golding model, these results could be replicated. 
Adding more bacteria-resource edges would introduce more competition, resulting in a higher rate of bacterial extinction and a decrease in community diversity. 
Removing edges would remove competition for resources, and community diversity would increase. 

\section{Other Users}
Although I created the simulation framework to facilitate the running and analysis of simulation results, the idea behind this program is that anyone can download the framework’s open-source code, edit it, and run their simulations. 
This tool could be used in an “Introduction to Bacteriology” or “Biological Modelling” course, where the professor would instruct the students to design and implement an ODE model. 
They would instruct the students to interact with the model as an introduction to modeling phage and bacteria populations. 
A researcher with weaker programming skills can utilize this low-code tool to gain a deeper understanding of how the system they are analyzing in their lab would behave under different conditions. 
The program is structured such that users with basic programming skills can create their own analyses by copying and pasting boilerplate-like code.
\Cref{AppendixD} contains a sample of the boilerplate code that implements the ODE model of the generalized Golding model.


Other users and researchers can program their own subset of tools to work with the primary tool.
They can use NetworkX to create the graph network programmatically.
Since the network is graph-based, they can run network analysis programs on it to identify the node with the most edges (most likely to be important and drive other population growths) and remove that node.
The user can create a tool that programmatically removes nodes and edges. 
Since the simulation framework is object-orientated, the user can interface with the framework from their code and skip the dashboard to automate every step along their simulation pipeline. 
This would enable users to run a broader range of more unique simulations programmatically and gain a greater understanding of phage-bacteria dynamics. 