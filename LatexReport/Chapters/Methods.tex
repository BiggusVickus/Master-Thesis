\subsection{Network Topography of Interactions}
In a microbial environment, there are numerous interactions between agents, but not every agent can and will interact with one another. Based on which agents interact with which agents, a network topography can be created, capturing the dynamics of the interactions. Each agent can be represented as a node. If an agent interacts with another agent, an edge can be linked between the agents. Each node can contain attributes and properties related to that agent, for example starting population or concentration, washout rate, or birth rate. Each edge likewise also contains attributes to capture the dynamic interactions between the agents, for example, resource usage, burst size, or affinity to infect. Adding the attributes to the nodes and edges allow for various The parameters can change between agents. For example, the initial population of phage 1 can be 300, while for phage 2 it is 150. Likewise, the attributes can be different between different agents. For example, bacteria 1 might use up resource 1 with rate constant 0.05, while bacteria 1 might use up resource 2 with rate constant 0.07. Bacteria 2 might not need resource 1 to survive, but bacteria 2 requires a lot of resource 2 to grow, with usage rate constant of 0.4. Using a graph network, these interactions between agents can be visualized, tracked and edited. \newline 

A tool has been developed to help aid in the development of this network topography. With this tool, a network topography can be created by adding any number of agents of varying type, such as bacteria, phages, or resources. There is an optional environment node that can capture global environment data, for example the length of the simulation, number of timesteps, temperature, pH, etc. The attributes of the agents, interactions, and environment can easily be edited. \newline 

Once a network topography capturing the interactions between any number of agents has been created, it would be useful to see how the population count or concentration value changes through time. 
A Python package has been created that allows for uploading a network topography, and with a small script that the user needs to provide, with the setting up of initial parameters and provided equations, runs a numerical solver using SciPy's solve\_ivp() function. 