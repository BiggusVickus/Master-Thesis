\subsection{Methods of Modelling Phages and Bacteria}
There are numerous ways to model the interactions between phages and bacteria.
Models can be built at a molecular level, where the model simulates the mechanical and chemical behavior of a phage as it interacts with the surface of a bacterium using computational chemistry methods.
On the other end of the spectrum, a different type of model can be built where populations of phages, bacteria, and resources can be modeled using Ordinary Differential Equations (ODEs) or Delay Differential Equations (DDEs).
DDEs are similar to ODEs, except where when ODEs are calculating the values of the equations at time $t$ using time $t-1$, DDEs can, but don't have to, use the value of the equation at time $t-\tau$, where $1 \leq \tau \leq t$. \newline 

Each type of system has its pros and cons.
With the molecular level model, the model is more complex and needs significantly more startup time, simulation time, and is in general much more complex.
However, more information can be gained from the simulations and can guide research in creating phages for a certain type of bacteria.
The ODE method is simpler and easier to set up, however it can only capture large population dynamics.
Certain assumptions about the community interactions have to be made.
For example, $\omega$ percent of the bacteria population is washed out.
The model can be made more complicated, by modelling each stage of the phage replication and lysis process, or instead of assuming exponential growth, there is a maximum carrying capacity of the population.
The model can be further altered by using a normally distributed variable $\textbf{N}(\mu=\omega, \sigma=1)$ to account for noise when measuring the data.
Ensuring the use of a seed value will ensure that each run of the model results in the same output. 

\subsubsection{Generalized Lotka-Voltera Model}
The Lotka-Voltera model, a first-order non-linear differential model, is a model that captures the dynamics between predators and prey, with phages being the predator and bacteria being the prey.
Any population can be modelled as such:
\[ 
    \frac{d{B}_i}{dt} = {B}_i \left(\left(r_i + \sum_{j}^{N} \alpha_{ij}{B}_j \right) - m_i\right)
\]
where ... 

\subsubsection{Generalized Consumer-Resource Model}
The generalized Consumer-Resource Model models the growth of a population and resource dynamics between a population of bacteria ${B}_i$ and a resource ${R}_i$. 
\begin{align}
    \frac{d{B}_i}{dt} &= r_i{B}_i \left(\sum_{\alpha} \Delta w_{i \alpha}C_{i \alpha}R_{\alpha}\right) - m_i {B}_i \\
    \frac{R_{\beta}}{dt} &= -\sum_i C_{i\beta}R_{\beta}{B_i} + \sum_{\alpha, i}D_{\beta\alpha}^{i}C_{i\alpha}R_{\beta}{B}_i \\
    \Delta w_{i\alpha} &= \sum_{\beta}D_{\beta \alpha}^{i}w_{\beta}
\end{align}

\subsubsection{Trait-Based Model}
The Trait-Based Model is a model that takes into account external factors such as the temperature or pH of the system and can be modeled as follows:  
\begin{align}
    \frac{dB_i}{dt} &= \left(r_i - m_i\right) B_i \\
    r_i &= \frac{r_{i\alpha}^{max}R_\alpha}{R_\alpha + K_{i\alpha}}e^{S_i\left(T-T_{ref}\right)}
\end{align}
where $S_i$ is the sensitivity to $B_i$ to factor $T$, and with trade off if $r_i^{max} > \text{ mean } r^{max} \text{ then } S_i > \text{ mean } S$. 

\subsubsection{Agent-Based Models}
Agent-based Models (ABM) model the system through space and time.
An $x \times y \times z$ grid (often $z$ is left out for a 2D system) is created and split into smaller subcells containing resources and microbes.
Each cell acts as its own tiny environment, where resources and microbes interact within the environment, but not with the neighboring cells.
Resources are diffused through the system using a PDE solver for a Boundary Value Problem (BVP).
Agents are allowed to move into neighboring grids with a probability $p$, where $p$ can depend on any number of parameters such as nutrient density, microbe density, or stochastic chance. \newline 
ABMs are useful when simulating many individual elements interacting in a system.
Chaotic or emergent behavior can arise from these interactions.
Chaotic behavior refers to the irregular and unpredictable evolution of a system's behavior due to nonlinear equations, exhibiting sensitive dependence on initial conditions \cite{encyclopedia_of_physical_science_and_technology}. \newline 
Emergent behavior is behavior that arises from the interactions of various agents in a system, that was not explicitly programmed into the system.
The behavior can be beneficial, neutral, or harmful, but it can not be predicted until it arises, \textit{if} it arises.
Agents can have simple rules, but when interacting with other agents, behavior that hasn't been programmed can arise.
Sometimes, people consider systems with emergent behaviors more complex than the sum of their parts. \newline
\begin{align} 
    \frac{\delta R_\alpha(r, t)}{\delta t} = \nabla \left[D \left( R_\alpha, r\right) \nabla R_\alpha \left( r, t \right) \right], r = \left(x, y\right)
\end{align}, where $r$ is a function of cell position $(x, y)$, and $t$ represents time. 
The cellular agents rules are as follows: 
\begin{align} 
    \frac{di}{dt} = r_i \left( \sum_\alpha \Delta w_{i\alpha}C_{i\alpha}R_\alpha\right)
\end{align}, where if $i> \text{ threshold, }\frac{i}{2}$ expands into the neighboring grid cell with a probability $p$. 
The system consumes resources and converts them into new sub-resource types with the following equation:
\begin{align}    
    \frac{dR_\alpha}{dt} &= -\sum_i C_{i\alpha}R_\alpha I \\
    \frac{dR_\beta}{dt} &= \sum_i C_{i\beta}R_\beta I + \sum_{\alpha, i}D_{\beta \alpha}^{i} C_{i \alpha} R_\alpha i
\end{align}. 

\subsection{Biology of Phages}