\chapter{Appendix A: Equation Parameters}
\label{AppendixA}
Parameters used in the various equations. 

\section{Simple/Advanced Golding Model Parameters}
\begin{table}[H]
    \small % or \footnotesize to make it more compact
    \centering
    \begin{tabularx}{\textwidth}{l l X}
        \toprule
        \textbf{Variable} & \textbf{Name} & \textbf{Description} \\
        \midrule
        $R / R_r$ & Resource entity & Resource $r$ concentration\\
        $U / U_b$ & Uninfected Bacteria population & Uninfected population for bacteria $b$ \\
        $I_i / I_{b_i}$ & Infected Bacteria population & Infected population for bacteria $b$ at stage $1, \dots, i, \dots, M$ \\
        $B / B_b$ & Bacteria population & Total bacteria population for bacteria $b$, assuming $B_b = U_b + \sum_{i=1}^M I_{b_i}$ \\
        $P / P_p$ & Phages population & Phage population for phage $p$ \\
        $e / e_{b r}$ & Consumption rate & Rate at which resource $r$ is consumed by bacteria $b$\\
        $\beta / \beta_{p b}$ & Burst size (B matrix)& Lytic burst size for phage $p$ and bacteria $b$\\
        $r / r_{p b}$ & Successful phage/cell encounter & Probability of a phage $p$ successfully infecting bacteria $b$\\
        $\tau / \tau_{b}$ & Latent period (tau vector)& Time it takes bacteria $b$ to go through the infection stage\\
        $v / v_{b r}$ & Maximal growth rate & Growth rate of bacteria $b$ from resource $r$ \\
        $K / K_{b r}$ & Monod Constant & Monod constant representing at what resource $r$ concentration at which bacteria $b$ grows at half its maximal rate $v$\\
        $\omega^i / \omega^i_r$ & wash-in rate & Rate of resource $r$ being added\\
        $\omega^o$ & wash-out rate & Rate of phages, bacteria, and resources being removed, acts on everything proportionally\\
        $M$ & Number of infection stages & Number of infection stages that a bacteria goes through, all bacteria entities have the same value for $M$\\
        $d$ & Debris & A debris term used to deactivate phages. See \Cref{sec:discussion:debris} for more information. \\
        $t$ & time & time value through the simulation \\
        \bottomrule
    \end{tabularx}
    \caption{
        Golding model parameters (\Cref{eq:golding_model} and \Cref{eq:adapted_golding_model}) with variables, names, and descriptions. 
        Subscripts on parameters indicate relationships; for example, $e_{b r}$ is nonzero if there is an edge connecting bacteria $b$ to resource $r$ in the network, zero otherwise.
    }
    \label{tab:appendixA:parameter_table_simple_golding_model}
\end{table}


\section{Sobol Parameters}
\begin{table}[H]
    \small % or \footnotesize to make it more compact
    \centering
    \begin{tabularx}{\textwidth}{l l X}
        \toprule
        \textbf{Variable} & \textbf{Name} & \textbf{Description} \\
        \midrule
        $Y$ & Univariate parameter output & univariate model output, such as mean $\mu$ or variance $\sigma$ \\
        $X$ & Input vector & Vector of size $d$, input vector to $f$ \\
        $i$ & Parameter input & Parameter $i$ of input \\
        $X_i$ & Parameter value & Value of vector $X$ at position $i=1, \dots, d$, the value of parameter $i$ \\
        $d$ & Input size & Size of input vector $X$ \\
        $X_{\sim i}$ & Parameter input & All values of $X$ that are not $X_i$ \\
        $f$ & Function $f$ & Arbitrary black-box function describing model \\
        $N$ & Samples & Number of samples, power of 2, $2^x$ \\
        $D$ & Parameter input size & Number of parameters inputted into Sobol, $d=|X|$ \\
        $ST_i$ & Global sensitivity & Contribution of parameter $X_i$ to output variance of $Y$ due to interactions with other variables \\
        $S1_i$ & First order sensitivity & Contribution of $X_i$ to output variance of $Y$ \\
        \bottomrule
    \end{tabularx} \newline
    \caption{
        Sobol parameter symbols, name, and description. 
        See \Cref{sec:Sobol_sensitivity_analysis} for the equations. 
    }
    \label{tab:appendixA:parameter_table_Sobol}
\end{table}

\section{Linear Regression Parameters}
\begin{table}[H]
    \small % or \footnotesize to make it more compact
    \centering
    \begin{tabularx}{\textwidth}{l l X}
        \toprule
        \textbf{Variable} & \textbf{Name} & \textbf{Description} \\
        \midrule
        $a$ & Slope & Slope of the linear regression line \\
        $c$ & Intercept & y-intercept of linear regression line\\
        $R^2$ & Regression Coefficient & Coefficient of determination of linear regression fit, quality of regression \\
        $x_i$ & Data point & Data point on the x-axis \\
        $y_i$ & Actual Value & Actual value of data for a given $x_i$\\
        $\hat{y_i}$ & Predicted Value & Value predicted of equation for a given $x_i$\\
        $\bar{y}$ & Average Value & Average $y$ value\\
        $n$ & Number of samples & Number of samples being tested\\
        \bottomrule
    \end{tabularx} \newline
    \caption{
        Variable symbol, name, and description used for the linear regression. 
    }
    \label{tab:appendixA:parameter_table_linear_regression}
\end{table}