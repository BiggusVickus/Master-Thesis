\chapter{Appendix C: Flowchart of User and System Interactions}
\label{AppendixC}

\Cref{fig:interaction_diagram} shows how the user can interact with the system, the input and outputs for subsystems, and the systems working with one another. 
 To read the flow chart, start from the top to the bottom. 
 First, the user creates a network using the GUI Network Creation Tool. 
 After the graph is complete, the user provides an implementation of the network as an ODE model in Python. 
 Once finished, the user provides the network file and ODE model to the ODE solver. 
 The solver uses information from the network file to determine the number of entities to create, parameter details (including names, values, and dimensions), and setting values.
 Then, the user interacts with the Visualization Dashboard Tool, for example, by clicking on buttons to run simulations, changing parameter values, (un)selecting checkboxes, zooming in and out of plots, and hovering over plots to show data. 
 Once a user has selected the parameter values, they are sent to the solver. 
 The solver calculates the time and population values using the provided graph and ODE model and then sends the data back to the Visualization Dashboard Tool, which outputs the visualizations. 
 If the user has run an Ultimate Analysis, they can query the saved data to create their custom visualizations.
\begin{figure}
    \centering
    \includegraphics[width=0.8\linewidth]{Images/interaction_diagram.pdf}
    \captionsetup{width=1\linewidth}
    \caption{
        The flowchart of user and system interactions. Read from top to bottom. 
    }
    \label{fig:interaction_diagram}
\end{figure} 