\chapter{Appendix F: Extra Plots and Figures}
\label{AppendixF}

\section{Extra SOBOL Analyses}
\label{sec:AppendixF:extra_SOBOL_analyses}

\begin{figure}[ht!]
    \centering
    \begin{subfigure}{0.49\linewidth}
        \centering
        \captionsetup{width=1\linewidth}
        \includegraphics[width=\linewidth]{Plots/Created/SOBOL/SOBOL_analysis_1748084143_Average.png}
        \caption{
            The $ST$ and $S1$ sensitivity for the average Resource, Uninfected, Infected, Phage, and Total Bacteria population. 
        }
        \label{fig:created:SOBOL_average_extra}
    \end{subfigure}
    \hfill
    \begin{subfigure}{0.49\linewidth}
        \centering
        \captionsetup{width=1\linewidth}
        \includegraphics[width=\linewidth]{Plots/Created/SOBOL/SOBOL_analysis_1748084143_Variance.png}
        \caption{
            The $ST$ and $S1$ sensitivity for the variance of the Resource, Uninfected, Infected, Phage, and Total Bacteria population. 
        }
        \label{fig:created:SOBOL_variance_extra}
    \end{subfigure}
    \caption{
        SOBOL analyses for the average, peak, and peak time. 
        The data was saved from the dashboard and plotted using Matplotlib. 
        The values used for this SOBOL test can be found in \Cref{tab:appendixE:SOBOL_analysis_values}. 
        The same data used in \Cref{fig:created:SOBOL_analyses} was used for \Cref{fig:created:SOBOL_average_extra} and \Cref{fig:created:SOBOL_variance_extra}. 
    }
    \label{fig:created:SOBOL_extra}
\end{figure}


\subsection{SOBOL Analysis Without Washin and Washout}
\begin{figure}[ht!]
    \centering
    \begin{subfigure}{0.32\linewidth}
        \centering
        \captionsetup{width=1\linewidth}
        \includegraphics[width=\linewidth]{Plots/Created/SOBOL/SOBOL_analysis_1748084143_Average.png}
        \caption{
            Final value, no washin and washout. 
        }
        \label{fig:created:SOBOL_final_no_wi_wo_extra}
    \end{subfigure}
    \hfill
    \begin{subfigure}{0.32\linewidth}
        \centering
        \captionsetup{width=1\linewidth}
        \includegraphics[width=\linewidth]{Plots/Created/SOBOL/SOBOL_analysis_1748084143_Average_(no_wi_and_wo).png}
        \caption{
            Average value, no washin and washout. 
        }
        \label{fig:created:SOBOL_average_no_wi_wo_extra}
    \end{subfigure}
    \hfill
    \begin{subfigure}{0.32\linewidth}
        \centering
        \captionsetup{width=1\linewidth}
        \includegraphics[width=\linewidth]{Plots/Created/SOBOL/SOBOL_analysis_1748084143_Variance_(no_wi_and_wo).png}
        \caption{
            Variance of population value, no washin and washout. 
        }
        \label{fig:created:SOBOL_variance_no_wi_wo_extra}
    \end{subfigure}
    \hfill
    \begin{subfigure}{0.32\linewidth}
        \centering
        \captionsetup{width=1\linewidth}
        \includegraphics[width=\linewidth]{Plots/Created/SOBOL/SOBOL_analysis_1748084143_Peak_(no_wi_and_wo).png}
        \caption{
            Peak population value, no washin and washout. 
        }
        \label{fig:created:SOBOL_peak_no_wi_wo_extra}
    \end{subfigure}
    \hfill
    \begin{subfigure}{0.32\linewidth}
        \centering
        \captionsetup{width=1\linewidth}
        \includegraphics[width=\linewidth]{Plots/Created/SOBOL/SOBOL_analysis_1748084143_Peak_Time_(no_wi_and_wo).png}
        \caption{
            Time of peak value, no washin and washout. 
        }
        \label{fig:created:SOBOL_peak_time_no_wi_wo_extra}
    \end{subfigure}
    \caption{
        SOBOL analyses for the final, average, variance, peak, and peak time, without a washin and washout rate.
        The data was saved from the dashboard and plotted using Matplotlib. 
        The values used for this SOBOL test can be found in \Cref{tab:appendixE:SOBOL_analysis_values}, except washin and washout have been set to 0. 
    }
    \label{fig:created:SOBOL_no_wi_wo_extra}
\end{figure}

\section{Why 95\%? }
\label{sec:appendixF:why_95}
The 95\% rule helps in the IVA analysis. 
Due to the solver, when taking the absolute peak value, the same time value can occur. 
Or in an ever increasing value like phages, the peak values occur at the last time step of the simulation, or plateaus and doesn't grow anymore. 
However, as the parameter value is changing, each graph for every input change will change the growth rate of the agent, changing how fast the agent population grows. 

\Cref{fig:appendixF:IVA_95_vs_100} shows how using the 95\% rule vs the 100\% rule for finding the max value reached helps smooth out computational errors from the ODE solver and smooths out the shape. 
For the phages, using the 100\% rule (\Cref{fig:appendixF:IVA_phages_100}) shows that the population peaked at the end of the simulation, $t=15$, for all $e$ values. 
However at $t=15$, the population plateaued, as evident by the line graph. 
Plotting the same plot, but calculating the peak at 95\% of the actual peak (\Cref{fig:appendixF:IVA_phage_95}) shows that the green line ($e=0.25$) "reached" its peak at $t=8.4$ before the red line ($e=0.05$) at $t=9.4$, a full unit of time after $e=0.25$. 
The user can thus conclude that for this instance, larger $e$ values will cause the phage population to reach its "peak" faster than smaller $e$ values. 

\Cref{fig:appendixF:IVA_uninfected_bacteria_95} and \Cref{fig:appendixF:IVA_uninfected_bacteria_100} likewise show how the 95\% rule can improve analysis of the change in peak time. 
\Cref{fig:appendixF:IVA_uninfected_bacteria_100} shows how apparently the peak is reached at set time values. 
Due to how \textit{solve\_ivp()} from SciPy works, it automatically chooses time values that it thinks would best capture the dynamics of the system without calculating too many steps. 
The user can control the step size by decreasing the absolute and relative error bounds, as well as by minimizing the time steps. 
The user can also provide their own time range with the number of steps to run, increasing the control of the time values chosen. 
It takes about 0.02321 seconds to run a simulation for 15 time units, where 200 time steps are selected and solved by the solver. 
Comparatively, a simulation with 1000 time units and 1000000 (a 5000x increase in samples) equidistant time samples takes about 1.71651 seconds to run, a 73.95562258x increase in time spent computing the simulation. 
The total time taken to run the whole method call, a call to the simple graph maker at the top of the dashboard took 1.76130 seconds vs 17.70634 seconds. 



Alternatively instead of controlling the solver, the user can use the 95\% rule.
Although some accuracy is lost. 
Going from the 100\% rule to the 95\% rule, the solver still captures the peak values and the dynamics, but the accuracy is lost. 
The 100\% rule shows that for $e=0.25$, the time the uninfected population reached its peak occurred at $t=3.2$. 
But for the 95\% rule, the time at which the peak occurred at is at $t=3.05$. 
The slope (the $a$ value) and the intercept (the $c$ value) are somewhat similar, with very high and similar $R^2$ values (0.97), suggesting a good linear fit of the data. 

\Cref{fig:appendixF:IVA_uninfected_bacteria_100_own_time} shows how the by increasing the time sampling to more fine grained results in a more accurate graph. 
Instead of having the solver choose the time values to test, 1000 equidistant time values were selected between 0 and 15. 
The solver can more accurately calculate the population values and calculate the proper peak time. 
Comparing the 100\% rule without the custom time values with the 100\% rule with the custom time values shows the same time values were calculated. 
in both, the $e=0.25$ resulted in a time of peak at 3.2 and for $e=0.05$, the time of peak occurred at $t=3.95$. 
This is in stark comparison to the 95\% rule vs 100\% rule without the custom time, showing a difference of $0.15$ time units. 
The custom time values also preserved the shape of the curve $e$-value vs time curve, being almost identical to that of the 95\% rule as seen in \Cref{fig:appendixF:IVA_uninfected_bacteria_95} and \Cref{fig:appendixF:IVA_uninfected_bacteria_100_own_time}. 

Another issue that arises with the custom time is that it doesn't solve the issue seen with the phages, where the time of peak is at $t=15$. 

The user can control the \% rule with a value input on the dashboard. 
They can select to use the 95\% rule, or 100\% rule, or even 83\% rule if they want by changing the value they use. 
The user can use their own custom time values, to ensure that they get high quality curves. 

\begin{figure}
    \centering
    \begin{subfigure}{1\linewidth}
        \centering
        \includegraphics[width=\linewidth]{Plots/Created/IVA/initial_value_analysis_Phages_95.png}
        \caption{
            IVA for phage population, 95\% rule
        }
        \label{fig:appendixF:IVA_phage_95}
    \end{subfigure}
    \hfill
    \begin{subfigure}{1\linewidth}
        \centering
        \includegraphics[width=\linewidth]{Plots/Created/IVA/initial_value_analysis_Phages_100.png}
        \caption{
            IVA for phage population, 100\% rule.  
        }
        \label{fig:appendixF:IVA_phages_100}
    \end{subfigure}
    \hfill
    \begin{subfigure}{1\linewidth}
        \centering
        \includegraphics[width=\linewidth]{Plots/Created/IVA/initial_value_analysis_Uninfected_Bacteria_95.png}
        \caption{
            IVA for uninfected population, 95\% rule. 
        }
        \label{fig:appendixF:IVA_uninfected_bacteria_95}
    \end{subfigure}
    \hfill 
    \begin{subfigure}{1\linewidth}
        \centering
        \includegraphics[width=\linewidth]{Plots/Created/IVA/initial_value_analysis_Uninfected_Bacteria_100.png}
        \caption{
            IVA for uninfected population, 100\% rule. 
        }
        \label{fig:appendixF:IVA_uninfected_bacteria_100}
    \end{subfigure}
    \begin{subfigure}{1\linewidth}
        \centering
        \includegraphics[width=\linewidth]{Plots/Created/IVA/initial_value_analysis_Uninfected_Bacteria_100_own_time.png}
        \caption{
            IVA for uninfected population, 100\% rule, 1000 equidistant time steps. 
        }
        \label{fig:appendixF:IVA_uninfected_bacteria_100_own_time}
    \end{subfigure}
    \caption{
        Testing the 95\% rule vs the 100\% rule, where the time at the absolute peak is taken and plotted in the second plot. 
        A comparison of phages and uninfected bacteria is shown. 
        Verification of the graph shape between the 95\% rule graph and a frequent time step with 100\% rule can be seen between c) and e). 
        The $e$ value is changed, ranging from 0.05 to 0.25. 
    }
    \label{fig:appendixF:IVA_95_vs_100}
\end{figure}

\section{Varying $r$ and $\beta$ In A $3\times 2\times 3$ System}
\Cref{fig:created:r_beta_washout_0}, \Cref{fig:created:r_beta_washout_0.02}, and \Cref{fig:created:r_beta_washout_0.05} show a 7x7 matrix of figures, each with a unique parameter set. 
In each sub-figure, the values of $r$ and $\beta$ are varied as follows: $r = 0.5, 1.1, 1.7, 2.3, 2.9, 3.5,$ inf; $\beta = 0, 20, 40, 60, 80, 100,$ inf. 
"inf" in the figures, represented as \textit{np.inf} in the code, the representation of $\infty$,  represents the original parameter values used in the IC, vector data, or matrix data. 
Each figure shows the effect of varying the washout rate, with values set to 0, 0.02, and 0.05, respectively.
All initial phage values were set to 10. 
This was done to really show how the different interactions with the bacteria, and the value of the parameters affected the phage growth. 
The default values for the parameters can be found in \Cref{tab:appendixE:complex_model}. 
\begin{figure}[]
    \includegraphics[width=1\textwidth]{Plots/Created/UA/r_beta_washout_0.png}
    \centering
    \caption{
        Washout $\omega^o=0$. 
    }
    \label{fig:created:r_beta_washout_0}
\end{figure}

\begin{figure}[]
    \includegraphics[width=1\textwidth]{Plots/Created/UA/r_beta_washout_0.02.png}
    \centering
    \caption{
        Washout $\omega^o=0.02$. 
    }
    \label{fig:created:r_beta_washout_0.02}
\end{figure}

\begin{figure}[]
    \includegraphics[width=1\textwidth]{Plots/Created/UA/r_beta_washout_0.05.png}
    \centering
    \caption{
        Washout $\omega^o=0.05$. 
    }
    \label{fig:created:r_beta_washout_0.05}
\end{figure}